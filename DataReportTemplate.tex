\documentclass[10pt, a4j]{jarticle} 

\usepackage[dvipdfmx]{graphicx} %図を挿入するためのパッケージ
\usepackage{amsmath,amssymb} %数学拡張パッケージ
\usepackage{layout} % \layoutと入れることでレイアウト情報が出る(印刷時には消すこと)
\usepackage{here} % tableで強制的に指定位置に配置するためのやつ.[H]と指定すると発動する.

%% ページ番号(現在ページ/総ページ)を右下に出す
\usepackage{fancyhdr}
\usepackage{lastpage}
\pagestyle{fancy}
\lhead{}
\rhead{}
\cfoot{}
\rfoot{\thepage{}/{}\pageref{LastPage}}
\renewcommand{\headrulewidth}{0pt}

%% 高さの設定
\setlength{\textheight}{\paperheight}   % ひとまず紙面を本文領域に
\setlength{\topmargin}{-5.4truemm}      % 上の余白を20mm(=1inch-5.4mm)に
\addtolength{\topmargin}{-\headheight}  % 
\addtolength{\topmargin}{-\headsep}     % ヘッダの分だけ本文領域を移動させる
\addtolength{\textheight}{-40truemm}    % 下の余白も20mmに
%% 幅の設定
\setlength{\textwidth}{\paperwidth}     % ひとまず紙面を本文領域に
\setlength{\oddsidemargin}{-5.4truemm}  % 左の余白を20mm(=1inch-5.4mm)に
\setlength{\evensidemargin}{-5.4truemm} % 
\addtolength{\textwidth}{-40truemm}     % 右の余白も20mmに


\begin{document} 

% <基礎実験フォーマット>
\renewcommand{\thefigure}{\thesection.\arabic{figure}} %図
\renewcommand{\thetable}{\thesection.\arabic{table}} %表
\renewcommand{\theequation}{\thesection.\arabic{equation}} %数式
% </ 基礎実験フォーマット>

%-------------------------
\section{実験結果} 
% <番号の振り直し>
\setcounter{figure}{0}
\setcounter{table}{0}
\setcounter{equation}{0}
% </ 番号の振り直し>

%-------------------------
\section{考察と結論} 
% <番号の振り直し>
\setcounter{figure}{0}
\setcounter{table}{0}
\setcounter{equation}{0}
% </ 番号の振り直し>

%-------------------------
\section{参考文献} 
% <番号の振り直し>
\setcounter{figure}{0}
\setcounter{table}{0}
\setcounter{equation}{0}
% </ 番号の振り直し>


\end{document} 